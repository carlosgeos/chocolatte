\documentclass[letterpaper]{article}
\usepackage{natbib}
\usepackage[utf8]{inputenc}

\usepackage{fullpage,epsf,fancyheadings}
\usepackage{amsmath}
\usepackage{graphicx}
\usepackage{tikz}

\usepackage{amsmath}
\usepackage[tocage]{appendix}
\usepackage[frenchb]{babel}
\usepackage[T1]{fontenc}
\usepackage{placeins, latexsym, amssymb}
\usepackage[hidelinks]{hyperref}
\usepackage{url}

\begin{document}
 \begin{titlepage}
 \begin{tikzpicture}[remember picture, overlay]
   \node [anchor=north east, inner sep=0pt]  at (current page.north east)
      {\includegraphics[height=3cm]{Baniere_ULB.png}};
 \end{tikzpicture}
 \begin{center}
 \textbf{\textsc{UNIVERSIT\'E LIBRE DE BRUXELLES}}\\
 \textbf{\textsc{Faculté des Sciences}}\\
 \textbf{\textsc{Département d'Informatique}}
 \vfill{}\vfill{}
 \begin{center}{\Huge INFO-F-302 Informatique Fondamentale: Rapport de projet}\end{center}{\Huge \par}
 \begin{center}{\large \textsc{Perale} Thomas\\\textsc{Requena} Carlos}\end{center}{\Huge \par}
 \vfill{}\vfill{}
 \vfill{}\vfill{}\enlargethispage{3cm}

 \begin{figure} [h!]
             \centering
     \includegraphics[width=4cm]{Sigle_ULB.png}
 \end{figure}

 \textbf{Année académique 2016~-~2017}
 \end{center}

 \end{titlepage}

\tableofcontents
\pagebreak

\section{Introduction}

Le but du projet est de modéliser des problèmes de satisfaction de contraintes
à l'aide de l'outil de résolution de contraintes \emph{Chocosolver}. Pour celà
on nous présente plusieurs problèmes en un premier temps basé sur un échiquier
et la modélisation des pièces de celui-ci. En un deuxième temps basé sur la
disposition d'un musée et de l'emplacement que doivent avoir les cameras dans
celui-ci.

\section{Question 1: Le problème d'indépendance}

\subsection{Définition du problème}

Comme il a été dit dans l'énoncé le problème d'indépendance consiste à
déterminer dans un échiquier de taille donné si il est possible d'assigner à
chacune des pièces misent à notre disposition (c'est à dire autant de
chevalier, tour ou fou que l'on veut) une position distincte de sorte qu'aucune
pièce ne menace une autre pièce.

\subsection{Définition des variables}

\begin{itemize}
   \item n la taille de l'échiquier;
   \item t le nombre de tour;
   \item f le nombre de fou;
   \item k le nombre de chevalier;
\end{itemize}

\subsection{Contraintes}

\section{Question 2: Le problème de domination}

\subsection{Définition du problème}

Ce problème cherche à ce que chaque case de l'échiquier doivent être soit
occupé soit menacée par au moins une pièce.

\subsection{Définition des variables}

\begin{itemize}
   \item n la taille de l'échiquier;
   \item t le nombre de tour;
   \item f le nombre de fou;
   \item k le nombre de chevalier;
\end{itemize}

\subsection{Contraintes}

\section{Question 4: Les chevaliers minimum}

\subsection{Définition du problème}

Ce problème calcule le nombre minimal de cavaliers permettant de dominer
un échiquier de taille donné.

\subsection{Définition des variables}

\begin{itemize}
   \item n la taille de l'échiquier;
\end{itemize}

\section{Question 5: La surveillance de musée}

\end{document}
